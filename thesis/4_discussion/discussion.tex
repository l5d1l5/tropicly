\chapter{Discussion and Conclusion}
\label{ch:discussion}
%TODO Research question
%TODO Deforestation mechanisms
%TODO Literature - compare your work to others
%TODO Strength - selling points
%TODO Limitations
%TODO Future research opportunitie

	\section{Software, design and technology}

	\section{Deforestation}
	\label{sec:discussion_deforestation}

		\subsection{Forest definition}
		\label{subsec:discussion_forest_definition}
			\begin{itemize}
				\item We can see at which regions Chen et al switched their tree cover definition to 10
				\item To improve we should apply for each tile a canopy class decision based on our analysis
				\item This could improve the improve the similarity (accuracy) by maximizing the sample count
				\item Use this jaccard method to exclude tiles where the tree cover similarity fall below a certain threshold
			\end{itemize}

		\subsection{Tree cover and deforestation}
		\label{subsec:discussion_tree_cover_and_deforestation}

		\subsection{Proximate deforestation driver}
		\label{subsec:discussion_proxy_deforestation_driver}
			\begin{itemize}
				\item Reclassification is not a good idea cause this approach leads not to consistent results
			\end{itemize}

		\subsection{Accuracy assessment}
		\label{subsec:discussion_accuracy_assessment}
			\begin{itemize}
				\item Is largely subjective because it is prepared from the study author
				\item Better if someone independent does it 
			\end{itemize}

	\section{Emissions}

	\section{Ecosystem service values}
		\begin{itemize}
			\item resilience of esv loss could be achieved over optimizing total value of the new land-use
			\item target optimization is use the clearcut by maximizing profit and minimizing the esv loss
		\end{itemize}

	\section{Binning analysis and visualization}
		\begin{itemize}
			\item Cut polygon by line the Scala (1992) approach explained with parametric separation function, and bezier
			\item A approach where ratio is also ratio of the hexagon area
		\end{itemize}

