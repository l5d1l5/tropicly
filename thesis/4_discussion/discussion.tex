\chapter{Discussion}
\label{ch:discussion}

	\section{Proximate deforestation drivers}
	\label{sec:discussion_deforestation}

		\subsection{Forest definition}
		\label{subsec:discussion_forest_definition}
			To harmonize the forest definition of the \ac{GL30} and \ac{GFC} dataset we developed a procedure based on \ac{JI} and statistical testing. By applying this approach we determined that the tree cover agreement between both strata is at its maximum within the canopy density interval $(10,100]$. At the global scale the agreement is approximately 65\% in median. We tested four different canopy density classes and excluded 10\% incremental per experiment group. However, we did not tested if a smaller steps size could yield differing results. Further, the upper canopy density threshold is 30\%. Therefore, we did not tested if the result changes if we exclude data with a denser canopy. However, it is not likely that canopy densities above 30\% lead to a increased agreement, due to the upper threshold of the \ac{GL30} dataset. Further, we tested the tree cover agreement for the three continental regions Latin America, Asia/Australia, and Africa. Latin America and Asia/Australia have the highest agreement, in median 70\% and 80\% at a canopy density interval of $(10,100]$. For Africa the agreement achieves only 40\% at a canopy density interval of $(10,100]$. The low tree cover agreement could be attributed to the tendency could be attributed to the tendency of the \ac{GFC} dataset to overestimate the canopy density for sparse woodland and figure \ref{fig:africa_tree_cover} shows that a large area is covered by sparse woodland in Africa \citep{Gross2017}. In regards to our methodology, the \ac{JI} have been criticized in scientific literature as metric to evaluate the accuracy of remotely sensed data \citep{Li2017a}. The critics are focused on the category focus of this index. It combines true positives (agreement), users and producers accuracy, while it omits true negatives (disagreement). Therefore, this metric puts a large weight on agreement. \citeauthor{Li2017a} argues that the mixing of producers and users accuracy obscure the information and overall accuracy, which weights true positives and true negatives equally is the preferable metric. However, this critic is derived from the application of the \ac{JI} for multi class remote sensing, while we use it for the accuracy assessment of a binary classification (forest and non-forest). A example shows the better performance of the \ac{JI} to scrutinize the agreement of two datasets in comparison to the overall accuracy: if true positives (agreement) and true negatives (disagreement) are equal sized and false negatives (producers accuracy), and false positives (users accuracy) are are zero both indexes equal to 1; if all properties (tp, tn, fn, fp) are equal sized the overall accuracy is 0.5, while the \ac{JI} is 0.33, if the agreement is zero and disagreement is some quantity the overall accuracy is 1 and \ac{JI} is 0. Therefore, if a metric is required to determine the agreement accuracy the \ac{JI} is the better choice. However, our method could be useful for other studies, that try to compare different tree cover datasets on its performance to predict forest cover. Till now several studies scrutinized the agreement of regional forest resource assessment with the global tree cover dataset \ac{GFC} \citep{Sannier2016,McRoberts2016,Gross2017}. If there is a general agreement between regional and global assessments, than this regions could improve carbon emissions reporting and monitoring of nature conservation progress without the large effort of national remote sensing programs. For Gabon \citeauthor{Sannier2016} questioned at which canopy density the \ac{GFC} dataset matches with the national forest inventory. They performed a accuracy assessment by considering different canopy densities but relayed on visual inspection for the selection of canopy density experiment groups. By applying our approach the selection could be automatized and future studies would gain an improved reasoning.

		\subsection{Tree cover and deforestation patterns}
		\label{subsec:discussion_tree_cover_and_deforestation}
			We scrutinized the tropical tree cover and deforestation patterns on regional, continental, and regional level. For the period 2001-2010 the global anthropogenic deforestation account for a tree cover loss of approximately 760314.7 km$^2$ on the global scale, while deforestation account for an area of approximately 388907, 196851, and 174555 km$^2$ in Latin America, Asia/Australia, and Africa, respectively. We applied our hexagonal binning to identify deforestation hot-spots for the three continental levels Latin America, Asia/Australia, and Africa. In Latin America the hot-spots are concentrated in Brazil, Paraguay, Argentina, Bolivia, Peru, and Colombia. For Asia/Australia our analysis revealed that deforestation hot-spots are located in Indonesia, Malaysia, Vietnam, and Laos. In Africa hot-spots are located in Angola, Democratic Republic of the Congo, Mozambique, Tanzania, Madagascar, and at the Ivory Coast. We compared our results with the literature, which showed similar results \ref{subsec:results_tree_cover_and_deforestation}.

		\subsection{Mapping of proximate deforestation driver}
		\label{subsec:discussion_proxy_deforestation_driver}
			We analyzed the spatial patterns and magnitude of \acp{PDD} of tropical tree cover at a regional, continental, and global scale between 2001-2010. Globally, the expansion of agricultural surfaces account for roughly 79.7\% (cropland, pastures, and regrowth aggregated as \citet{Geist2001} suggests) of the tree cover loss. Natural changes which corresponds to the class wetland and water account for 1.8\% of the forest loss, while the expansion of artificial surfaces account for 0.4\% of the deforestation. Further, approximately 18.4\% of the the forest loss can be attributed to other causes like bareland or shrubland. A rigorous comparison of our results with other global estimates from literature is complex because the methodology, time periods, extent, and classification schema differ largely. For 2000-2010 \citet{Hosonuma2012} estimates that agriculture, mining, infrastructure, and urbanization account for approximately 82\%, 8\%, 8\%, and 2\% of the tree cover loss in 100 tropical and subtropical countries. Our estimate of agricultural expansion corresponds to \citeauthor{Hosonuma2012}. For the entire global forest \citet{Curtis2018} estimates that commodity-driven deforestation, shifting agriculture, forestry, wildfires, and urbanization account for 25\%, 21\%, 31\%, 22\%, <1\% of the tree cover loss for the period 2001-2015. The estimates of this study can only compared in regards to urbanization, where we get the same results. In Latin America the most dominant cause for deforestation is the expansion of pastures, which accounts for 40.8\% of the forest loss. For 1990-2005 \citet{Sy2015} estimates that roughly 71.2\% of the forest loss can be attributed to pasture expansion in South America. The large difference to our study can be attributed to the time frame and methodology. In our study pasture expansion dominated the forest transitions at the deforestation hot-spots in Brazil, Colombia, and Guatemala. Deforestation hot-spots located in Paraguay, Argentina, and Bolivia are prevailed by cropland expansion. In Asia/Australia the major \ac{PDD} are regrowth dynamics, which account for 61.2\% of the forest loss respectively. This large quantity of regrowth dynamics can be attributed to the large area covered by tree crops in Asia especially in Malaysia and Indonesia \citep{Corley2016,Austin2019}. The deforestation hot-spots in Indonesia and Malaysia are dominated by regrowth dynamics, which could be the establishment of new tree crops or the rotational cycle as management practice. In Laos and Vietnam cropland conversion are the dominant cause of forest loss at the hot-spots. For Africa the major \ac{PDD} is the expansion of pastures, which accounts for 46\% of th tree cover loss. On country level the causes of deforestation are varying largely, while \citet{Curtis2018} estimates that 92\% of the tree cover loss can be attributed to shifting agriculture in Africa. However, our results and several regional focused studies show that relevant deforestation for stable \ac{LC} transitions occur \citep{Ruf2014,Kideghesho2015,Barima2016,Folefack2019}. At the Ivory Coast the deforestation hot-spots are dominated by cropland and regrowth dynamics, while in DR Congo the hot-spots are exposed to pasture expansion. In Angola the expansion of cropland is the major \ac{PDD} at the deforestation hot-spots, while Mozambique's and Tanzania's deforestation hot-spots are exposed to cropland and grassland expansion. For Madagascar we observed large regrowth dynamics, which could be attributed to shifting agriculture. The described dynamics at the deforestation hot-spots are largely confirmed by the literature as section \ref{subsec:results_proxy_deforestation_drivers} describes. Our method is promising candidate for a quick evaluation of \acp{PDD} on different ranges. However, for a generous evaluation of the results secondary literature of data is required to deduce more specific dynamics of deforestation. Further, this approach can not explain more complex proximate deforestation dynamics like the degradation of forest cover by fuelwood collection and logging followed by transitions to cropland, which is a common dynamic in Africa \cite{Geist2001,Cabral2011}. A major advantage of our method is, that a long support cycle is scheduled for the two datasets \ac{GL30} and \ac{GFC}. For the latter dataset a new version is released each annum. The next version of \ac{GL30}dataset should target the global \ac{LC} at the year 2015, and it is in discussion to release it on a regularly basis \citep{Chen2017}. Further, it is in discussion to increase the class size for the next release of the dataset. Therefore, long term studies on \acp{PDD} dynamics with a better class resolution are possible in future. Further, to derive spatial patterns of \acp{PDD} and to present the results we developed our own visualization approach, a hexa-pie-chart choropleth cartogram. This approach can present multi-variate data in a comprehensive and reasonable manner as the maps in section \ref{subsec:results_proxy_deforestation_drivers} show. However, improvements for our approach could be: modification of the pie-chart split algorithm and the representation of ratios as a pieces of the hexagon. In case of the split algorithm an enhancement could be the method described by \citet{Skala1994}. A modified version of the method based on parametric separation functions could be used to generalize the pie-chart generation for the entire set of convex polygons. In case of enhancing the representation of ratios as a pieces of the hexagon: Now, we determine the piece size as a ratio of the y-range but hexagons have not a equal area under its curve at each point of y like quadrilaterals. To enhance this a scaling approach could be used, where we draw within the hexagon interior a new polygon scaled down by a ratio.

		\subsection{Accuracy assessment}
		\label{subsec:discussion_accuracy_assessment}
			We performed an accuracy assessment for our \acp{PDD} mapping product derived from the \ac{GL30} and \ac{GFC} dataset. The product has an overall accuracy of approximately 75\%, while the most accurate classes in terms of producers accuracy are cropland, regrowth, grassland, and shrubland (producers accuracy ranges between 71\% and 81\%). For the remaining \ac{LC} classes forest, wetland, water, artificial, and bareland the producers accuracy ranges between 36\% and 67\%. The low accuracy within this classes can be attributed to our reclassification approach. The approach favors classes which are more frequent over the entire dataset for the reclassification. This can be explained by probability theory: if construct the buffer around a cluster of misclassified pixels the most dominant \ac{LC} classes are more likely to appear within the buffer. The reclassification should be modified to nearest neighbor approach, where the reassignment is determined by the nearest neighboring cluster of a \ac{LC} class. Further, the accuracy assessment should be repeated by applying the approach described by \citet{Olofsson2014}. By applying this we would gain for each \acp{PDD} class a uncertainty assessment for the derived area estimates. This would be an improvement for the computation of derived features like the \ac{ESV} and emission estimates. However, the accuracy assessment was performed by the author but for this kind of assessments it is pivotal that it is executed by independent experts.

	\section{Carbon emissions}

	\section{Ecosystem service values}
		For the tropical zone we analyzed the \ac{ESV} dynamics on a global and continental level by using benefit transfer. We derived the \ac{ESV} unit value for different biomes from three different commonly used datasets: \citet{Costanza2014} (Co), \citet{Groot2012} (Dg), and \citet{Siikamaki2015} (Wb). Tropical tree cover loss between 2001-2010 account for an \ac{ESV} loss of of 414.1 (Co), 402.2 (Dg), or 101.1 (Wb) billion dollar per year for the three datasets, respectively. \citet{Song2018} estimates by applying Co's unit values that tropical tree cover loss accounts for a loss of 550.7 billion dollar per year on a global range. The difference from our estimate can be attributed to variations in the usage of the \ac{GFC} dataset. For the study \ac{ESV} loss is computed by considering tree cover loss within the entire canopy density range. Latin America has the largest loss of \ac{ESV} by tropical deforestation followed by Asia/Australia and Africa. This can be attributed to the large deforestation in Latin America between 2001 and 2010. In regards of differences between the \ac{ESV} estimates of the three datasets. The unit values of \citet{Groot2012} related to the same database as the \citet{Costanza2014} values only form an earlier state. Therefore, this unit values can be replaced by the \citet{Costanza2014} dataset. The Co dataset cover the most biomes and is to favor if \ac{ESV} dynamics are scrutinized for a wide variety of \ac{LC} types. The dataset of \citet{Siikamaki2015} has the smallest unit value for forest but it provides for each country a distinct estimate. The difference of the unit value for forest between Co and Wb can be attributed to the valuation of ecosystem services and the considered number of sub-services. Therefore, both estimates prepared by both datasets are not comparable. The \ac{ESV} net loss account for -47.3 (Co), -121.5 (Dg), or -43.7 (Wb) billion dollar per year on a global range. In Asia/Australia the smallest net loss is noticed followed by Africa and Latin America. The difference is related to \ac{LC} transition patterns and total forest loss. In Asia/Australia most of the forest loss is driven by regrowth dynamics, while the total forest loss is smallest in Africa. To best of our knowledge no estimates for \ac{ESV} dynamics by \acp{PDD} are provided in literature for a generous comparison with our study. However, the net loss estimates show that the \ac{ESV} loss is by magnitudes smaller if we consider the transitions of forest land to other \ac{LC} classes This reveals a fundamental problem of applying \ac{ESV} unit values to \ac{LC} change dynamics. If the entire forest cover would be replaced by cropland or artificial surfaces the net loss is positive, which deduces to a gain in \ac{ESV}. Further, the coverage of different biomes by unit values is not diverse enough. In Asia/Australia and the other continents we had to use the unit value of tropical forest to compute the \ac{ESV} value of regrowth. However, often regrowth is the establishment of tree crops or plantations, which not corresponds to the \ac{ESV} of natural forests. Therefore, a greater variety of unit values is needed to cover different \ac{LC} scenarios.