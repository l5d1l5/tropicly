\chapter{Discussion}
\label{ch:discussion}
	\section{Proximate deforestation drivers}
	\label{sec:discussion_deforestation}

		\subsection{Forest definition}
		\label{subsec:discussion_forest_definition}
			To harmonize the forest definition of the \ac{GL30} and \ac{GFC} dataset we developed a procedure based on \ac{JI} and statistical testing. By applying this approach we determined that the tree cover agreement between both strata is at its maximum within the canopy density interval $(10,100]$. At the global scale the agreement is approximately 65\% in median. We tested four different canopy density classes and excluded 10\% incremental per experiment group. However, we did not tested if a smaller steps size could yield differing results. Further, the upper canopy density threshold is 30\%. Therefore, we did not tested if the result changes if we exclude data with a denser canopy. However, it is not likely that canopy densities above 30\% lead to a increased agreement, due to the upper threshold of the \ac{GL30} dataset. Further, we tested the tree cover agreement for the three continental regions Latin America, Asia/Australia, and Africa. Latin America and Asia/Australia have the highest agreement, in median 70\% and 80\% at a canopy density interval of $(10,100]$. For Africa the agreement achieves only 40\% at a canopy density interval of $(10,100]$. The low tree cover agreement could be attributed to the tendency could be attributed to the tendency of the \ac{GFC} dataset to overestimate the canopy density for sparse woodland and figure \ref{fig:africa_tree_cover} shows that a large area is covered by sparse woodland in Africa \citep{Gross2017}. In regards to our methodology, the \ac{JI} have been criticized in scientific literature as metric to evaluate the accuracy of remotely sensed data \citep{Li2017a}. The critics are focused on the category focus of this index. It combines true positives (agreement), users and producers accuracy, while it omits true negatives (disagreement). Therefore, this metric puts a large weight on agreement. \citeauthor{Li2017a} argues that the mixing of producers and users accuracy obscure the information and overall accuracy, which weights true positives and true negatives equally is the preferable metric. However, this critic is derived from the application of the \ac{JI} for multi class remote sensing, while we use it for the accuracy assessment of a binary classification (forest and non-forest). A example shows the better performance of the \ac{JI} to scrutinize the agreement of two datasets in comparison to the overall accuracy: if true positives (agreement) and true negatives (disagreement) are equal sized and false negatives (producers accuracy), and false positives (users accuracy) are are zero both indexes equal to 1; if all properties (tp, tn, fn, fp) are equal sized the overall accuracy is 0.5, while the \ac{JI} is 0.33, if the agreement is zero and disagreement is some quantity the overall accuracy is 1 and \ac{JI} is 0. Therefore, if a metric is required to determine the agreement accuracy the \ac{JI} is the better choice. However, our method could be useful for other studies, that try to compare different tree cover datasets on its performance to predict forest cover. Till now several studies scrutinized the agreement of regional forest resource assessment with the global tree cover dataset \ac{GFC} \citep{Sannier2016,McRoberts2016,Gross2017}. If there is a general agreement between regional and global assessments, than this regions could improve carbon emissions reporting and monitoring of nature conservation progress without the large effort of national remote sensing programs. For Gabon \citeauthor{Sannier2016} questioned at which canopy density the \ac{GFC} dataset matches with the national forest inventory. They performed a accuracy assessment by considering different canopy densities but relayed on visual inspection for the selection of canopy density experiment groups. By applying our approach the selection could be automatized.

		\subsection{Tree cover and deforestation patterns}
		\label{subsec:discussion_tree_cover_and_deforestation}
			\begin{itemize}
				\item We analyzed the tree cover and deforestation patterns on global, continental, and regional scale between 2000 and 2010.
				\item Between 2000 and 2010 anthropogenic deforestation account for a tree cover loss of 760314.7 km$^2$ at an global level.
				\item Deforestation account for an area of approximately of 388907, 196851, and 174555 km$^2$ the tree cover loss in Latin America, Asia/Australia, and Africa, respectively.
				\item By using our hexagonal binning approach we identified deforestation hot-spots for the three continental regions Latin America, Asia/Australia, and Africa.
				\item In Latin America deforestation hot-spots are located in Brazil, Paraguay, Argentina, Bolivia, Peru, and Colombia.
				\item In Asia/Australia deforestation hot-spots are located in Indonesia, Malaysia, Vietnam, and Laos.
				\item In Africa deforestation hot-spots are located in Angola, Democratic Republic of the Congo, Mozambique, Tanzania, Madagascar and at the Ivory Coast.
			\end{itemize}

		\subsection{Mapping of proximate deforestation driver}
		\label{subsec:discussion_proxy_deforestation_driver}
			\begin{itemize}
				\item We analyzed the spatial patterns and magnitude of PDDs at an regional, continental and global scale by aggregating the most recent global \ac{LC} datasets \ac{GL30} and \ac{GFC} for the time frame 2000 till 2010.
				\item On a global scale agricultural changes account for 79.7\% (cropland, pastures, and regrowth) of the deforestation.
				\item Natural changes account for 1.8\% of the forest loss, while artificial changes account for 0.4\% of the forest loss. Other use causes account for 18.4\% of the forest change.
				\item Rigorous comparing to other studies is not feasible because the methodology, period, and classification schema differs largely. 
				\item In Latin America the dominant cause of deforestation is the expansion of pastures (40.8\%). 
				\item This driver dominated the the forest change at the deforestation hot-spots of Brazil, Colombia, and Guatemala.
				\item Cropland expansion is the major deforestation driver for the hot-spots in Paraguay, Argentina, and Bolivia.
				\item In Asia/Australia the dominant proximate driver of deforestation are regrowth dynamics.
				\item This can be attributed to the high amount of tree crops in Malaysia and Indonesia.
				\item The deforestation hot-spots in Indonesia and Malaysia are dominated by regrowth dynamics, which could be the establishment of new tree crops and the rotational cycle as management practice.
				\item The deforestation hot-spots in Laos and Vietnam are dominated by cropland expansion.
				\item In Africa the dominant cause of forest loss is the expansion of pastures (46\%).
				\item The causes of deforestation on country level show a large variation in Africa.
				\item At the Ivory Cost the hot-spots are dominated by cropland expansion and regrowth dynamics. The causes are explained in the results section.
				\item The deforestation hot-spots in the Democratic Republic of the Congo are dominated by pasture expansion.
				\item In Angola the expansion of cropland is the major \ac{PDD} in the deforestation hot-spots.
				\item In Mozambique and Tanzania causes for deforestation at the hot-spots are cropland and grassland expansion.
				\item The deforestation hot-spots in Madagascar are dominated by regrowth dynamics which could be attributed to shifting agriculture \citep{Curtis2018}.
				\item Our developed approach is a promising candidate for a quick evaluation of \acp{PDD} on different scalings.
				\item However, the approach needs data from additional sources to evaluate the \ac{LC} transitions to regrowth.
				\item Our approach can not explain complex interactions between multiple \acp{PDD} like degradation of forest cover by fuelwood collection and logging followed by a transition to cropland, which is common in Africa \citep{Geist2001,Cabral2011}.
				\item A major advantage of our approach is that it is to expect that the datasets \ac{GFC} and \ac{GL30} have a long support.
				\item \ac{GFC} appears each annum a new version.
				\item The next version \ac{GL30} should appear for the year 2015 and it is in discussion to release the datasets on a regularly basis \citep{Chen2017}.
				\item Further, the classification schema will be improved by adding more classes to the classification.
				\item Improvements consider deforestation only for a minimal mapping unit to exclude scattering.
				\item To present spatial patterns we developed our own visualization approach.
				\item This approach can present multivariate data in a comprehensive and accessible way.
				\item There are still improvements which can be done: the split algorithm can be modified and the pie chart needs a revisit.
				\item In case of the split algorithm the line clipping approach described by \citep{Skala1994} could be used in a modified version to generalize our pie chart generation for the entire set of convex polygons.
				\item A separation function can be used to determine if the point is above or under the split line and then the intersection can be determined.
				\item A large problem is the relative are representation. We use the y range to determine the point were we split the hexagon.
				\item But hexagon have no equal are like squares, therefore 25\% are not 25\% of the hexagon area.
				\item To solve this we could use a scaling representation
			\end{itemize}

		\subsection{Accuracy assessment}
		\label{subsec:discussion_accuracy_assessment}
			We performed an accuracy assessment for our \acp{PDD} mapping product derived from the \ac{GL30} and \ac{GFC} dataset. The product has an overall accuracy of approximately 75\%, while the most accurate classes in terms of producers accuracy are cropland, regrowth, grassland, and shrubland (producers accuracy ranges between 71\% and 81\%). For the remaining \ac{LC} classes forest, wetland, water, artificial, and bareland the producers accuracy ranges between 36\% and 67\%. The low accuracy within this classes can be attributed to our reclassification approach. The approach favors classes which are more frequent over the entire dataset for the reclassification. This can be explained by probability theory: if construct the buffer around a cluster of misclassified pixels the most dominant \ac{LC} classes are more likely to appear within the buffer. The reclassification should be modified to nearest neighbor approach, where the reassignment is determined by the nearest neighboring cluster of a \ac{LC} class. Further, the accuracy assessment should be repeated by applying the approach described by \citet{Olofsson2014}. By applying this we would gain for each \acp{PDD} class a uncertainty assessment for the derived area estimates. This would be an improvement for the computation of derived features like the \ac{ESV} and emission estimates. However, the accuracy assessment was performed by the author but for this kind of assessments it is pivotal that it is executed by independent experts.

	\section{Emissions}

	\section{Ecosystem service values}
		For the tropical zone we analyzed the \ac{ESV} dynamics on a global and continental level by using benefit transfer. We derived the \ac{ESV} unit value for different biomes from three different commonly used datasets: \citet{Costanza2014} (Co), \citet{Groot2012} (Dg), and \citet{Siikamaki2015} (Wb). Tropical tree cover loss between 2000-2010 account for an \ac{ESV} loss of of 414.1 (Co), 402.2 (Dg), or 101.1 (Wb) billion dollar per year for the three datasets, respectively. \citet{Song2018} estimates by applying Co's unit values that tropical tree cover loss accounts for a loss of 550.7 billion dollar per year on a global range. The difference from our estimate can be attributed to variations in the usage of the \ac{GFC} dataset. For the study \ac{ESV} loss is computed by considering tree cover loss within the entire canopy density range. Latin America has the largest loss of \ac{ESV} by tropical deforestation followed by Asia/Australia and Africa. This can be attributed to the large deforestation in Latin America between 2000 and 2010. In regards of differences between the \ac{ESV} estimates of the three datasets. The unit values of \citet{Groot2012} related to the same database as the \citet{Costanza2014} values only form an earlier state. Therefore, this unit values can be replaced by the \citet{Costanza2014} dataset. The Co dataset cover the most biomes and is to favor if \ac{ESV} dynamics are scrutinized for a wide variety of \ac{LC} types. The dataset of \citet{Siikamaki2015} has the smallest unit value for forest but it provides for each country a distinct estimate. The difference of the unit value for forest between Co and Wb can be attributed to the valuation of ecosystem services and the considered number of sub-services. Therefore, both estimates prepared by both datasets are not comparable. The \ac{ESV} net loss account for -47.3 (Co), -121.5 (Dg), or -43.7 (Wb) billion dollar per year on a global range. In Asia/Australia the smallest net loss is noticed followed by Africa and Latin America. The difference is related to \ac{LC} transition patterns and total forest loss. In Asia/Australia most of the forest loss is driven by regrowth dynamics, while the total forest loss is smallest in Africa. To best of our knowledge no estimates for \ac{ESV} dynamics by \acp{PDD} are provided in literature for a generous comparison with our study. However, the net loss estimates show that the \ac{ESV} loss is by magnitudes smaller if we consider the transitions of forest land to other \ac{LC} classes This reveals a fundamental problem of applying \ac{ESV} unit values to \ac{LC} change dynamics. If the entire forest cover would be replaced by cropland or artificial surfaces the net loss is positive, which deduces to a gain in \ac{ESV}. Further, the coverage of different biomes by unit values is not diverse enough. In Asia/Australia and the other continents we had to use the unit value of tropical forest to compute the \ac{ESV} value of regrowth. However, often regrowth is the establishment of tree crops or plantations, which not corresponds to the \ac{ESV} of natural forests. Therefore, a greater variety of unit values is needed to cover different \ac{LC} scenarios.