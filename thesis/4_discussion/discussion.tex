\chapter{Discussion and Conclusion}
\label{ch:discussion}
%TODO Research question
%TODO Deforestation mechanisms
%TODO Literature - compare your work to others
%TODO Strength - selling points
%TODO Limitations
%TODO Future research opportunitie

	\section{Software, design and technology}

	\section{Deforestation}
	\label{sec:discussion_deforestation}

		\subsection{Forest definition}
		\label{subsec:discussion_forest_definition}
			\begin{itemize}
				\item For a regional approach a better solution could be to select for each region independently the right canopy density. For America a good agreement between the tree cover could be achieved by selecting the second class. For Asia by selecting the first class and for africa the second. Even better would be to decide per tile individually which canopy density should be selected. This would eliminate regional effects of different forest densities.
				\item discuss regions independently asia and america have large tree cover agreement
				\item africa has the lowest agreement only core forest zones show high similarity
				\item We can see at which regions Chen et al switched their tree cover definition to 10
				\item To improve we should apply for each tile a canopy class decision based on our analysis
				\item This could improve the improve the similarity (accuracy) by maximizing the sample count
				\item Algorithm draft for single similarity: Compute jaccard indexes for tile pair at different canopy densities, put results in a list, sort the list in decreasing order, pick the class where jaccard index is max
				\item Use this jaccard method to exclude tiles where the tree cover similarity fall below a certain threshold
			\end{itemize}

		\subsection{Tree cover and deforestation}
		\label{subsec:discussion_tree_cover_and_deforestation}

		\subsection{Proximate deforestation driver}
		\label{subsec:discussion_proxy_deforestation_driver}
			\begin{itemize}
				\item Reclassification is not a good idea cause this approach leads not to consistent results
				\item limitation the conversion of tree cover to shrubland or grassland in Africa can be mapping error
			\end{itemize}

		\subsection{Accuracy assessment}
		\label{subsec:discussion_accuracy_assessment}
			\begin{itemize}
				\item Is largely subjective because it is prepared from the study author
				\item Better if someone independent does it 
				\item Even better if you have ground truth prepared by field studies
				\item Class variability errors source the reclassification
				\item Time frame of our classification we classified our ground truth data at google image data
				\item Land cover change dynamics stay hidden
				\item Sample size is not scientific chosen
				\item Confusion matrix neglect uncertainty, apply approach in olofsson et al should improve the reasoning
				\item mention that kappa coefficient should be neglected cause issues reference olofsson 
			\end{itemize}

	\section{Emissions}

	\section{Ecosystem service values}
		\begin{itemize}
			\item resilience of esv loss could be achieved over optimizing total value of the new land-use
			\item target optimization is use the clearcut by maximizing profit and minimizing the esv loss
			\item large differences between the datasets
		\end{itemize}

	\section{Binning analysis and visualization}
		\begin{itemize}
			\item Cut polygon by line the Scala (1992) approach explained with parametric separation function, and bezier
			\item A approach where ratio is also ratio of the hexagon area
		\end{itemize}

