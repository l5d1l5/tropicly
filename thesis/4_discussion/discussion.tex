\chapter{Discussion}
\label{ch:discussion}
	\section{Proximate deforestation drivers}
	\label{sec:discussion_deforestation}

		\subsection{Forest definition}
		\label{subsec:discussion_forest_definition}
			\begin{itemize}
				\item We developed an Procedure to harmonize the forest definition between two different datasets based on Jaccard Index and statistical testing.
				\item We determined that the tree cover agreement between gfc and gl30 is at its maximum at 10\% with a median agreement of 65\% on the global level.
				\item Whereas we did not tested if a smaller steps size could change this result or if canopy densities above 30\% yield better results.
				\item Whereby above 30\% is not likely that the agreement increase because the gl30 data has there its upper threshold.
				\item At an continental level the tree cover agreement between gfc and gl30 was largest in asia and americas (70 and 80 median), while africa has the smallest agreement.  
				\item The low agreement in Africa could be attributed to the tendency of gfc to overestimate tree cover in sparse woodland \citep{Gross2017}.
				\item Jaccard index has category focus and combines users and producers accuracy, while it omits true negatives. Therefore it puts large weight on agreement.
				\item \citep{Li2017a} argues that the mixing of producers and users accuracy obscure information and overall accuracy which weights bot equally is to prefer as a metric.
				\item But \citep{Li2017a} used the jaccard index for multi class problems, while we use it for binary classification.
				\item Further, we require a metric which focuses the agreement and not consider agreement in disagreement.
				\item As an example could show why the jaccard index is better for tree cover agreement: if tp and tn are equal and fp and fn are zero both indexes give 1, if all are equal sized overall accuracy is 0.5 and jaccard index is 0.33, if the agreement is zero and disagreement is n then overall accuracy is 1 and jaccard index is 0.
				\item For the case if the agreement of tree cover must be highlighted this index is far better than overall accuracy.
				\item Therefore our research could be useful for other studies which try to compare different tree cover datasets on their performance.
				\item There are several studies which compare country data with global tree cover data \citep{Sannier2016,McRoberts2016,Gross2017}.
				\item Because if the datasets match country definition this countries can effort better reporting of carbon emissions, monitor progress of nature conservation etc. without the effort of an own national remote sensing programm.
				\item As example \citep{Sannier2016} tried to determine at which canopy density gfc matches the country dataset of Gabon.
				\item They performed a accuracy assessment which different canopy densities but they relayed on visual inspection to determine which canopy density they should select.
				\item With the our approach it could be automatized.
				\item A algorithm proposal for smaller units or regional optimization: compute jaccard indexes for selected canopy density intervals and pick the greatest
				\item Extent the method by carefully applied accuracy assessment of clustering of disagreement 
			\end{itemize}

		\subsection{Tree cover and deforestation patterns}
		\label{subsec:discussion_tree_cover_and_deforestation}
			\begin{itemize}
				\item Improved tree cover map divide tree cover within hexagon by landmass within hexagon
				\item Improved loss map divide loss by the tree cover within a hexagon
			\end{itemize}

		\subsection{Mapping of proximate deforestation driver}
		\label{subsec:discussion_proxy_deforestation_driver}
			\begin{itemize}
				\item We analyzed the spatial patterns and magnitude of PDDs at an regional, continental and global scale.
				\item compare to other studies
				\item 
				\item To present spatial patterns we developed our own visualization approach.
				\item This approach can present multivariate data in a comprehensive and accessible way.
				\item There are still improvements which can be done: the split algorithm can be modified and the pie chart needs a revisit.
				\item In case of the split algorithm the line clipping approach described by \citep{Skala1994} could be used in a modified version to generalize our pie chart generation for the set of convex polygons.
				\item A separation function can be used to determine if the point is above or under the split line and then the intersection can be determined.
				\item A large problem is the relative are representation. We use the y range to determine the point were we split the hexagon.
				\item But hexagon have no equal are like squares, therefore 25\% are not 25\% of the hexagon area.
				\item To solve this we could use a scaling representation
			\end{itemize}

		\subsection{Accuracy assessment}
		\label{subsec:discussion_accuracy_assessment}
			\begin{itemize}
				\item Should be done by independent person was not the case
				\item Should be improved by using approach discussed in \citep{Olofsson2014}
				\item This approach would add uncertainties in area estimates
			\end{itemize}

	\section{Emissions}

	\section{Ecosystem service values}
		\begin{itemize}
			\item We estimated the esv dynamics on a global and continental level by using benefit transfer. 
			\item We used three different datasets to do so.
			\item Tropical tree cover loss account for an \ac{ESV} loss of 414.1, 405.2, and 101.1 billion dollar per year for co, dg, and wb, respectively.
			\item \citep{Song2018} estimates that tree cover loss account for an loss 550.7 billion dollar per year by applying the unit values of costanza.
			\item The difference can be attributed that song used the entire gfc dataset to compute the esv of forest loss.
			\item In regards of the differences between the estimates of the three datasets.
			\item \citep{Groot2012} unit values are from the same database as \citep{Costanza2014} only from a earlier state.
			\item Therefore this unit values can be replaced by the values of \citep{Costanza2014} and must not considered for other studies.
			\item \citep{Costanza2014} is the dataset which covers the most biomes.
			\item This dataset is to favor to account esv changes for forest cover changes because a large number of change biomes can be considered by the aggregation. 
			\item The dataset of \citep{Siikamaki2015} has the smallest unit value for forest.
			\item The large difference between the global unit values can be attributed to differences in ecosystem services they consider
			\item Therefore the estimated of co and wb are not comparable.
			\item The net esv loss account for -47.3, -121.5, and -43.7 billion dollar per year.
			\item As this values show is the loss of esv by magnitudes smaller if we consider the transition of forest land to other land cover types.
			\item Due to lack of other studies performed these kind of analysis we can not compare our results.
			\item Till know the esv are computed only for global land cover change dynamics for all land cover classes.
			\item This reveals a fundamental problem of applying esv estimates to land cover change dynamics.
			\item If the entire forest cover is replaced by cropland or artificial than the net balance is positive.
			\item This means the new land cover would have a greater esv than the old forest cover.
			\item Further problematic is the handling of secondary forest, tree crops.
			\item If we stick to biomes they must be computed with the forest class but this is questionable.
			\item Therefore a greater variety of esv unit values is needed for more diverse biomes. 
		\end{itemize}
