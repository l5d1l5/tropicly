\chapter{Conclusion}
\label{ch:conclusion}
%TODO one page
Classification schema:

Grassland clearing of tree cover for grassland indicates that it is used for pastures \note{there is somewhere a citation for it}. Water and wetland class can be assumed that it is not urgent a anthropogenic driven tree cover loss. It could be forest loss by inundation by lakes and rivers.\cite{Sy2015}
Shrubland is a ambivalent class, could be abandoned pastures, newly established tree plantations or cropland. Regrowth is introduced by the \ac{GFC} gain layer, this are areas where tree cover regrowth is detectable. This could be tree plantations or natural regrowth. Here secondary literature is needed to distinguish between these terms. It depends largely on regions and what regrowth is. Areas still classified as forest are ambivalent it could be a false positive by \ac{GFC} loss layer (predicts loss but there is no loss), a false negative by gain layer (predicts no gain but there is gain), a tree cover definition discrepancy between \ac{GFC} and \ac{GL30} respectively remote sensing method differences.

Therefore we will aggregate the classes by the following conventions for the discussion of the results. To cropland and grassland we will refer as agricultural land. Regrowth and forest are forestry and plantation activities. Shrubland will be discussed extra. Artificial is artificial and bareland, wetland, and 