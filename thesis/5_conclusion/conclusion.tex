\chapter{Conclusion}
\label{ch:conclusion}
	The expansion of agricultural land is the major reason for forest cover loss in the tropical zone for the 2001-2010 time period. However, the agricultural type as deforestation driver differs per continental zone and regional scale. Therefore, to encounter deforestation on policy level each country must justify its efforts individually. Further, our study demonstrate that existing remotely-sensed datasets are viable to produce reasonable data for \acp{PDD}. This, could be advantageous for countries without own remote sensing programs on monitoring \ac{LC} change and enable them to surveil nature conservation efforts and the adaption of new policy frameworks. During this study we experimented with the harmonization of tree cover agreement between to distinct \ac{LC} datasets by applying the \ac{JI}. Our presented approach could help countries to harmonize their local forest cover datasets with the global datasets. \acp{PDD} are a major contributor of carbon loss in terms of biomass removal in the tropical region. However, our study demonstrate that the carbon loss by \ac{SOC} change can't be underestimated and must be considered for carbon accounting. Finally, we scrutinized as first in the literature the impact of \acp{PDD} on the \ac{ESV} dynamics where our findings show that focusing only on \ac{ESV} loss by forest cover loss can be misleading. Only if the value of \ac{LC} transitions is quantified the net balance can explain if an action results in a loss or gain of \ac{ESV}. However, this demonstrates that the \ac{ESV} unit values require a careful readjustment to not overestimate the value of cropland and urban in comparison with tropical forest. Further, a \ac{ESV} valuation covering more distinct biomes is required to present more detailed quantifications on global and local range. As future task we identified the requirement of an extensive uncertainty assessment for all three features we evaluated.

% missing
%Abstract
%	german
%	polish