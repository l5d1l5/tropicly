\thispagestyle{empty}

\begin{de}
	\begin{center}
		\textbf{Zusammenfassung}
	\end{center}
	\textbf{Titel: Quantifizierung der Triebkräft der tropischen Entwaldung und ihre Auswirkung auf die Kohlenstoffvorräte sowie Ökosystemdienstleistungen}

	In den Tropen ist Entwaldung einer der Hauptursachen für die Veränderung der Bodenbedeckung. Das kann auf verschiedene Ursachen zurückgeführt werden, die als direkte Triebkräfte der Entwaldung bezeichnet werden. Weltweit ist die fortschreitende Entwaldung eine Quelle für anthropogene Treibhausgas-Emissionen (THG) und trägt somit zum Klimawandel bei. Außerdem werden globale und kontinentale Ökosytemdiestleistungen in ihrer Funktionsweise durch Veränderungen der Waldstruktur gestört. Die räumliche Auswertung von Entwaldungstriebkräften ist aufwending und bestehende Datensätze sind nur eingeschränkt in niedriger Auflösung verfügbar. Weiterhin wurden THG-Emissionen nur für Biomasseverluste quantifiziert, während Kohlenstoffverluste durch Veränderungen des Bodengefüges nicht bewertet wurden. Schlußendlich wurde bisher nur der Bruttoverlust von Ökosystemdienstleistungen qunatifiziert ohne die Nettoverluste mit einzubeziehen. Ziel dieser Studie ist es die folgenden Dynamiken für den Zeitraum 2001-2010 zu analysieren: (1) die räumliche und quantitative Verteilung von Entwaldungstriebkräften, (2) die damit verbundenen Kohlenstoffverluste durch Biomasseverluste und Veränderung des Bodengefüges (3) die Wertentwicklung der Ökosystemdienstleistungen im globalem, kontinentalem, und regionalem Maßstab. Durch Aggregation der hochauflösenden Fernerkundungsdatensätze GlobeLand30 und Global Forest Change haben wir die Triebkräfte der Entwaldung quantifiziert. Die Bilanzierung der Kohlenstoffverluste durch die Entfernung von Biomasse erfolgte über einen Biomassedichtedatensatz. Um die Kohlenstoffverluste durch Veränderungen des Bodengefüges zu bestimmen haben wir die Koeffizienten von \citet{Don2010} verwendet. Die Bilanz der Ökosystemdienstleistungen wurde mit hilfe der drei gängisten Koeffizienten Datensätze erstellt. In den Tropen ist Landwirtschaft (79.7\%) der Hauptgrund für Entwaldung, während Expansion von Weideflächen (33.1\%) den größten Anteil daran hat. Weiterhin konnten wir feststellen das der Gesamtverlust von Kohlenstoff durch Entfernung von Biomasse 6757 Mt C beträgt, während der Kohlenstoffverlust durch Veränderung des Bodengefüges 583 ($\pm$ 105) Mt C beträgt. Der Bruttorverlust an Ökosystemdienstleistungen durch Entwaldung beträgt 414.1 Milliarden Dollar, während der Nettoverlust bei 63.1 Milliarden Dollar pro Jahr liegt. Unsere Ergebnisse zeigen das die bereits verfügbaren fernerkundungsdatensätze geeignet sind um Entwalungsdynamiken zu beschreiben sowie die Kohlenstoffverluste und Ökosystemdienstleistungen zu quantifizieren. Unsere studie zeigt das direkte Entwaldungstriebkräfte einen wesentlichen Beitrag zu den globalen Emissionen und dem Werteverlust in Ökosystemdienstleistungen beitragen.

	Schlüsselwörter: Entwaldung, direkte Triebkräfte, THG Emissionen, Ökosystemdienstleistungen, Tropen
\end{de}