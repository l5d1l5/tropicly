\thispagestyle{empty}

\begin{center}
	\textbf{Summary}
\end{center}
	\textbf{Title: Global assessment of deforestation drivers across the tropics: impacts on carbon stocks and ecosystem services}

	Deforestation is the major type for land cover change in the tropical region. These land cover changes relate to distinct causes recognized as direct or proximate deforestation drivers. Globally, deforestation constitutes a major source of anthropogenic \ac{GHG} emissions, thereby contributing substantially to climate change. Additionally, changes in tropical forest cover have significant impacts on global and continental ecosystem services. The spatially-explicit evaluation of proximate deforestation drivers requires a large effort, while existing products are rare and only available in coarse resolution. Further, \ac{GHG} emissions are quantified only in terms of biomass removal, while the quantification of carbon losses by soil organic carbon change is lacking. Additionally, the impact of deforestation drivers on ecosystem services are only assessed in regards to losses without any closer picture on the dynamics. Therefore, this study aims to analyze for the 2001-2010 time period: (1) the magnitudes of deforestation drivers through a spatially-explicit approach, (2) the related carbon losses resulting from the removal of biomass and soil organic carbon changes, and (3) the ecosystem service value dynamics of tropical deforestation on a global, continental, and regional scale. By aggregating the most recent high-resolution land cover datasets GlobeLand30 and Global Forest Change we quantified the proximate drivers of deforestation. Carbon loss resulting from biomass removal is derived from an aboveground live woody biomass density map.  For the evaluation of carbon losses resulting from soil organic carbon changes we applied the GSOCmap and the soil organic carbon change coefficients from \citet{Don2010}. Finally, to assess the impact of proximate deforestation drivers on ecosystem services the three most common unit value datasets are used. On a global scale agriculture (79.7\%) represents the major deforestation driver, while the expansion of pastures contributed most (33.1\%). The continental scaled analysis revealed that proximate deforestation driver dynamics differ in magnitude and spatial distribution. In the last decade the total carbon loss by biomass removal and soil organic carbon content change accounted for 6757 Mt C and 583 ($\pm$ 105) Mt C, respectively. Furthermore, tropical deforestation contributed during this period to a gross ecosystem service value loss of 414.1 billion dollar per year, while the net loss represented 63.5 billion dollar per year.  Our findings demonstrate that existing remotely-sensed datasets are viable to assess deforestation dynamics and to quantify derived features like carbons losses and ecosystem dynamics. In addition, they demonstrate that deforestation drivers represent a major contribution to global carbon emissions and ecosystem service value loss.

	Keywords: deforestation, proximate driver, GHG emissions, ecosystem services, tropics
