\thispagestyle{empty}

\begin{center}
	\textbf{Summary}
\end{center}
	\textbf{Title: Tropical deforestation and its impact on land cover, carbon losses, and ecosystem service values}

	Deforestation is the major cause for land cover change in the tropical zone. These land cover changes relate to distinct direct causes recognized as direct/proximate deforestation drivers. Forest land change by proximate causes is a large source of CO$_2$, which is a major contributor to human induced climate change. Further, the change of tropical forest cover has a significant impact on global and continental ecosystem services, which are a pivotal element for climate change mitigation. Till now the spatial explicit evaluation of proximate deforestation drivers requires a large effort, while existing products are only available in coarse resolution. Further, carbon emissions are quantified only in terms of biomass removal, while the quantification of carbon losses by soil organic carbon change is still lacking. Additionally, the impact of deforestation drivers on ecosystem services are only assessed in regards to losses without any closer picture on the dynamics. Therefore, this study aims to analyze: (1) spatial explicitly the magnitudes of proximate deforestation driver, (2) the related carbon losses by the removal of biomass and soil organic carbon changes, and (3) the ecosystem service value dynamics of tropical deforestation on a global, continental, and regional scale between 2001-2010. By aggregating the most recent high-resolution land cover datasets GlobeLand30 and Global Forest Change we quantified the proximate drivers of deforestation \citep{Hansen2013,Chen2015}. Carbon loss by biomass removal is derived from the Aboveground live woody biomass density map \citep{Baccini2015}. We applied the GSOCmap and soil organic carbon change coefficients from \citet{Don2010} for the evaluation of carbon losses by soil organic carbon change \citet{FAO2018}. To assess the impact of proximate deforestation drivers on ecosystem services the three most common unit value datasets from \citet{Costanza2014}, \citet{Groot2012}, and \citet{Siikamaki2015} are used. On global scale agriculture (79.7\%) is the major deforestation driver, while the expansion of pastures contributed most (33.1\%). The continental scaled analysis revealed that proximate deforestation driver dynamics differ in magnitude and spatial distribution. The gross carbon loss by biomass removal and soil organic carbon content change accounts for 6757 Mt C and 583 ($\pm$ 105) Mt C, respectively. Tropical deforestation contributes to an ecosystem service value loss of 414.1 billion Int'I\$ y$^{-1}$, while the net balance accounts for -63.5 billion Int'I\$ y$^{-1}$. Our findings demonstrate, that existing remotely sensed datasets are viable to assess deforestation dynamics and to quantify derived features like carbons losses and ecosystem dynamics. In addition, they demonstrate that ecosystem service unit values requires an adjustment.

	Keywords: tropics, deforestation, emissions, ecosystem, hexagon, jaccard