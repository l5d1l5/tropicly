\thispagestyle{empty}

\begin{center}
	\textbf{Summary}
\end{center}
	\textbf{Title: Global assessment of deforestation drivers across the tropics: impacts on carbon stocks and ecosystem services}

	Deforestation is the major cause for land cover change in the tropical region. These land cover changes relate to distinct causes recognized as direct/proximate deforestation drivers. Deforestation is a major source of anthropogenic \ac{GHG} emissions, which contributes to climate change. Further, the change of tropical forest cover has a significant impact on global and continental ecosystem services. The spatial explicit evaluation of proximate deforestation drivers requires a large effort, while existing products are only available in coarse resolution. Further, \ac{GHG} emissions are quantified only in terms of biomass removal, while the quantification of carbon losses by soil organic carbon change is lacking. Additionally, the impact of deforestation drivers on ecosystem services are only assessed in regards to losses without any closer picture on the dynamics. Therefore, this study aims to analyze: (1) spatial explicitly the magnitudes of deforestation drivers, (2) the related carbon losses by the removal of biomass and soil organic carbon changes, and (3) the ecosystem service value dynamics of tropical deforestation on a global, continental, and regional scale between 2001 and 2010. By aggregating the most recent high-resolution land cover datasets GlobeLand30 and Global Forest Change we quantified the drivers of deforestation. Carbon loss by biomass removal is derived from an aboveground live woody biomass density map. We applied the GSOCmap and soil organic carbon change coefficients from \citet{Don2010} for the evaluation of carbon losses by soil organic carbon change. To assess the impact of proximate deforestation drivers on ecosystem services the three most common unit value datasets are used. On global scale agriculture (79.7\%) is the major deforestation driver, while the expansion of pastures contributed most (33.1\%). The continental scaled analysis revealed that proximate deforestation driver dynamics differ in magnitude and spatial distribution. In the last decade the total carbon loss by biomass removal and soil organic carbon content change accounts for 6757 Mt C and 583 ($\pm$ 105) Mt C, respectively. Tropical deforestation contributes to an ecosystem service value gross loss of 414.1 billion dollar per year, while the net loss accounts for 63.5 billion dollar per year. 
	Our findings demonstrate, that existing remotely sensed datasets are viable to assess deforestation dynamics and to quantify derived features like carbons losses and ecosystem dynamics. In addition, they demonstrate that deforestation driver are a major contributor to carbon emissions. Further, they demonstrate that ecosystem service value accounting must be considered in terms of loss and gain to receive net changes. Finally, we outline why the unit values require an valuation adjustment.

	Keywords: tropics, deforestation, emissions, ecosystem, hexagon, jaccard