\thispagestyle{empty}

\begin{de}
	\begin{center}
		\textbf{Zusammenfassung}
	\end{center}
	\textbf{Titel: Quantifizierung der Auslöser der tropischen Entwaldung und ihre Auswirkung auf die Kohlenstoffvorräte sowie Ökosystemdienstleistungen}

	In den Tropen ist Entwaldung eine der Hauptursachen für die Veränderung der Bodenbedeckung. Das kann auf verschiedene Ursachen zurückgeführt werden, die als direkte Auslöser der Entwaldung bezeichnet werden. Weltweit ist die fortschreitende Entwaldung eine Quelle für anthropogene Treibhausgasemissionen (THGE) und trägt somit zum Klimawandel bei. Außerdem werden durch Veränderungen der Waldstruktur globale und kontinentale Ökosytemdiestleistungen in ihrer Funktionsweise  gestört. Die räumliche Bewertung von Entwaldungsauslösern ist aufwending und bestehende Datensätze sind nur eingeschränkt und in niedriger Auflösung verfügbar. Weiterhin wurden THGE nur für Biomasseverluste quantifiziert, während Kohlenstoffverluste durch Veränderungen des Bodengefüges nicht bewertet wurden. Schlußendlich wurde bisher nur der Bruttoverlust von Ökosystemdienstleistungen quantifiziert ohne die Nettoverluste mit einzubeziehen. Ziel dieser Studie ist es die folgenden Dynamiken für den Zeitraum 2001-2010 zu analysieren: (1) die räumliche und quantitative Verteilung von Entwaldungsauslösern, (2) die damit verbundenen Kohlenstoffverluste durch Biomasseverluste und Veränderung des Bodengefüges, (3) die Wertentwicklung der Ökosystemdienstleistungen in globalem, kontinentalem, und regionalem Maßstab. Durch Aggregation der hochauflösenden Fernerkundungsdatensätze GlobeLand30 und Global Forest Change wurden Auslöser der Entwaldung quantifiziert. Die Bilanzierung der Kohlenstoffverluste durch die Beseitigung von Biomasse erfolgte über einen Datensatz zur Biomassedichte. Um die Kohlenstoffverluste durch Veränderungen des Bodengefüges zu bestimmen, wurden die Koeffizienten von \citet{Don2010} verwendet. Die Bilanz der Ökosystemdienstleistungen wurde mit hilfe der drei gängisten Koeffizienten Datensätze erstellt. In den Tropen ist Landwirtschaft (79.7\%) der Hauptgrund für Entwaldung, während Expansion von Weideflächen (33.1\%) den größten Anteil daran hat. Weiterhin wurde festgestellt, dass der Gesamtverlust von Kohlenstoff durch Beseitigung von Biomasse 6757 Mt C beträgt, während der Kohlenstoffverlust durch Veränderung des Bodengefüges 583 ($\pm$ 105) Mt C beträgt. Der Bruttorverlust an Ökosystemdienstleistungen durch Entwaldung beträgt 414.1 Milliarden Dollar, während der Nettoverlust bei 63.1 Milliarden Dollar pro Jahr liegt. Die Ergebnisse zeigen, dass die bereits verfügbaren Fernerkundungsdatensätze geeignet sind um Entwalungsdynamiken zu beschreiben sowie die Kohlenstoffverluste und Ökosystemdienstleistungen zu quantifizieren. Die Studie zeigt, dass direkte Entwaldungsauslöser einen wesentlichen Beitrag zu den globalen THGE und dem Werteverlust in Ökosystemdienstleistungen beitragen.

	Schlüsselwörter: Entwaldung, direkte Auslöser, THGE, Ökosystemdienstleistungen, Tropen
\end{de}