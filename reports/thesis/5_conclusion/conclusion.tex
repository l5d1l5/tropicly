\chapter{Conclusion}
\label{ch:conclusion}
	The expansion of agricultural land is the major reason for forest cover loss in the tropical zone for the 2001-2010 time period. However, the agricultural type as deforestation driver differs per continental zone and regional scale. Therefore, each country must adjust individually its policy efforts to successfully reduced deforestation. Further, our study demonstrates that existing remotely-sensed datasets are viable to produce reasonable data for \acp{PDD}. This could be advantageous for countries without own remote sensing programs on monitoring \ac{LC} change and could enable them to surveil nature conservation efforts and the adoption of new policy frameworks. During this study we experimented with the harmonization of tree cover agreement between different \ac{LC} datasets by applying the \ac{JI}. Our presented approach could help countries to harmonize their local forest cover datasets with the global datasets. \acp{PDD} are a major contributor of carbon loss in terms of biomass removal in the tropical region. However, our study demonstrates that the carbon loss resulting from \ac{SOC} change can substantially underestimate results and should therefore be considered in carbon accounting methodologies. Finally, we scrutinized for the first time the impact of \acp{PDD} on the \ac{ESV} dynamics. Our findings show that focusing only on \ac{ESV} loss by forest cover loss can be misleading. Only if the value of \ac{LC} transitions is quantified can the net balance explain if an action results in a loss or gain of \ac{ESV}. However, this demonstrates that the \ac{ESV} unit values require a careful readjustment so as to not overestimate the value of cropland and urban area in comparison with tropical forest. Further, a \ac{ESV} valuation covering more distinct biomes is required to present more detailed quantifications on a global and local range. As the required data becomes available, future research should consider including uncertainty assessments in the three features.