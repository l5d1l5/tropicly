\chapter{Discussion}
\label{ch:discussion}

	\section{Proximate deforestation drivers}
	\label{sec:discussion_deforestation}
		Based on our first research question we developed an approach to scrutinize the spatially-explicit \acp{PDD} of tropical forest cover loss for the period 2001 till 2010. We relied for the mapping of deforestation causes on the most recent \ac{LC} datasets \ac{GL30} and \ac{GFC}. To successfully extract the \acp{PDD} from both products we harmonized the forest definition between them, derived tree cover and deforestation patterns, determined the magnitudes of different \acp{PDD}, and performed an accuracy assessment. In this section we will discuss each mentioned sub-objective separately by starting with the tree cover harmonization. After, we discuss the deforestation patterns followed by the mapping of \acp{PDD}. Finally, we will discuss the accuracy assessment. 

		To harmonize the forest definition of the \ac{GL30} and \ac{GFC} dataset we developed a procedure based on the \ac{JI} and statistical testing. By applying this approach we determined that the tree cover agreement between both strata is at its maximum within the canopy density interval $(10,100]$. At the global scale the median tree cover agreement is approximately 65\%. We tested four different canopy density classes and excluded 10\% increments per experiment group. However, we did not tested if a smaller steps size could yield differing results. Further, the upper canopy density threshold is 30\%. Therefore, we did not test if the result changes if we exclude data with a denser canopy. However, it is not likely that canopy densities above 30\% lead to a increased agreement, due to the upper threshold of the \ac{GL30} dataset. 

		Further, we tested the tree cover agreement for the three continental regions of Latin America, Asia/Australia, and Africa. Latin America and Asia/Australia have the highest agreement, in median 70\% and 80\% at a canopy density interval of $(10,100]$. For Africa the agreement achieves only 40\% at a canopy density interval of $(10,100]$. This low tree cover agreement could be attributed to the tendency of the \ac{GFC} dataset to overestimate the canopy density for sparse woodland. Figure \ref{fig:africa_tree_cover} demonstrate that a large area is covered by sparse woodland in Africa \citep{Gross2017}. Our results show, that the tree cover detection for Africa must be improved and validated. Most remote sensing technologies perform good at homogeneous \ac{LC} types but fail to produces reliable results if the \ac{LC} type gets inhomogeneous.

		In regards to our methodology, the \ac{JI} has been criticized in the scientific literature as a metric to evaluate the accuracy of remotely sensed data \citep{Li2017a}. The critics mention the category focus of this index. It combines true positives (agreement), users and producers accuracy, while it omits true negatives (disagreement). Therefore, this metric puts a large weight on agreement. \citet{Li2017a} argues that the mixing of producers and users accuracy obscures the information. However, this critic is derived from the application of the \ac{JI} for multi class remote sensing, while we use it for the accuracy assessment of a binary classification (forest and non-forest). A example shows the better performance of the \ac{JI} to scrutinize the agreement of two datasets in comparison to the overall accuracy: if true positives (agreement) and true negatives (disagreement) are equally sized and false negatives (producers accuracy) and false positives (users accuracy) are zero both indexes equal to 1; if all properties (tp, tn, fn, fp) equal sized the overall accuracy is 0.5, while the \ac{JI} is 0.33, if the agreement is zero and disagreement is some quantity the overall accuracy is 1 and \ac{JI} is 0. Therefore, if a metric is required to determine the agreement accuracy the \ac{JI} is the better choice. 

		However, our method could be useful for other studies that try to compare different tree cover datasets on its performance to predict forest cover. Until now several studies scrutinized the agreement of regional forest resource assessment with the global tree cover dataset \ac{GFC} \citep{Sannier2016,McRoberts2016,Gross2017}. If there is a general agreement between regional and global assessments, than these regions could improve carbon emissions reporting and monitoring of nature conservation progress without the large effort of national remote sensing programs. \citet{Sannier2016} questioned at which canopy density the \ac{GFC} dataset matches with the national forest inventory in Gabon. They performed an accuracy assessment by considering different canopy densities but relied on visual inspection for the selection of canopy density experiment groups. By applying our approach the selection could be automatized and future studies would gain an improved reasoning.

		We scrutinized the tropical tree cover and deforestation patterns on a regional, continental, and regional level. For the period 2001-2010 the global anthropogenic deforestation accounted for a tree cover loss of approximately 760314.7 km$^2$ on the global scale, while deforestation accounted for an area of approximately 388907 km$^2$, 196851 km$^2$, and 174555 km$^2$ in Latin America, Asia/Australia, and Africa, respectively. We applied our hexagonal binning to identify deforestation hotspots for the three continental levels of Latin America, Asia/Australia, and Africa. In Latin America the hotspots are concentrated in Brazil, Paraguay, Argentina, Bolivia, Peru, and Colombia. For Asia/Australia our analysis revealed that deforestation hotspots are located in Indonesia, Malaysia, Vietnam, and Laos. In Africa hotspots are located in Angola, Democratic Republic of the Congo, Mozambique, Tanzania, Madagascar, and in Ivory Coast. We compared our results with the literature, which showed similar results (section \ref{subsec:results_tree_cover_and_deforestation}).

		We analyzed the spatial patterns and magnitude of \acp{PDD} of tropical tree cover at a regional, continental, and global scale between 2001-2010. Globally, the expansion of agricultural surfaces accounted for roughly 79.7\% (cropland, pastures, and regrowth aggregated as \citet{Geist2001} suggests) of the tree cover loss. Natural changes which corresponds to the wetland and water classes accounted for 1.8\% of the forest loss, while the expansion of artificial surfaces represented for 0.4\% of the deforestation. Further, approximately 18.4\% of the the forest loss can be attributed to other \ac{LC} types like bareland or shrubland. A rigorous comparison of our results with other global estimates from the literature is complex because the methodology, time periods, extent, and classification schema differ largely. For 2000-2010 \citet{Hosonuma2012} estimated that agriculture, mining, infrastructure, and urbanization accounted for approximately 82\%, 8\%, 8\%, and 2\% of the tree cover loss in 100 tropical and subtropical countries. Our estimate of agricultural expansion corresponds to \citet{Hosonuma2012}. For the entire global forest \citet{Curtis2018} estimates that commodity-driven deforestation, shifting agriculture, forestry, wildfires, and urbanization account for 25\%, 21\%, 31\%, 22\%, <1\% of the tree cover loss for the period 2001-2015. The estimates of this study can only compared in regards to urbanization, where we get the same results. Our study shows, that agriculture is the largest deforestation driver but a continental analysis is required to derive more detailed results. 

		In Latin America the most dominant cause for deforestation is the expansion of pastures, which accounts for 40.8\% of the forest loss. For 1990-2005 \citet{Sy2015} estimated that 71.2\% of the forest loss can be attributed to pasture expansion in South America. The large difference to our study can be attributed to the time frame and methodology. In our study pasture expansion dominated the forest transitions at the deforestation hotspots in Brazil, Colombia, and Guatemala. Deforestation hotspots located in Paraguay, Argentina, and Bolivia are predominantly driven by cropland expansion. 

		In Asia/Australia the major \ac{PDD} are regrowth dynamics, which account for 61.2\% of the forest loss respectively. This large quantity of regrowth dynamics can be attributed to the large area covered by tree crops in Asia, especially in Malaysia and Indonesia \citep{Corley2016,Austin2019}. The deforestation hotspots in Indonesia and Malaysia are dominated by regrowth dynamics, which likely represent the establishment of new tree crops or the rotational cycle as management practice. In Laos and Vietnam cropland conversion were the dominant cause of forest loss at the hotspots. 

		For Africa the major \ac{PDD} is the expansion of pastures, which accounts for 46\% of the tree cover loss. On a country level the causes of deforestation are vary largely, while \citet{Curtis2018} estimates that 92\% of the tree cover loss can be attributed to shifting agriculture in Africa. However, our results and several regional focused studies show that relevant deforestation for stable \ac{LC} transitions do occur \citep{Ruf2014,Kideghesho2015,Barima2016,Folefack2019}. In the Ivory Coast the deforestation hotspots are dominated by cropland and regrowth dynamics, while in DR Congo the hotspots are exposed to pasture expansion. In Angola the expansion of cropland is the major \ac{PDD} at the deforestation hotspots, while Mozambique's and Tanzania's deforestation hotspots are exposed to cropland and grassland expansion. For Madagascar we observed large regrowth dynamics, which could be attributed to shifting agriculture. The described dynamics at the deforestation hotspots are largely confirmed by the literature as section \ref{subsec:results_proxy_deforestation_drivers} describes. 

		Our analysis shows that the types of agriculture causes of deforestation differ between the continental regions. Our method represents a promising candidate for a quick evaluation of \acp{PDD} at different scales. However, for a generous evaluation of the results secondary literature of data is required to deduce more specific dynamics of deforestation. Further, this approach can not explain more complex proximate deforestation dynamics like the degradation of forest cover by fuelwood collection and logging followed by transitions to cropland, which is a common dynamic in Africa \citep{Geist2001,Cabral2011}. A major advantage of our method is that a long support cycle is scheduled for the two datasets \ac{GL30} and \ac{GFC}. For the latter dataset a new version is released per annum. The next version of the \ac{GL30} dataset should target the global \ac{LC} at the year 2015, and it is in discussion to release it on a regularly basis \citep{Chen2017}. Further, it is in discussion to increase the number of \ac{LC} classes for the next release of the dataset. Therefore, long term studies on \acp{PDD} dynamics with a better class resolution might be possible in future. 

		Further, to derive spatial patterns of \acp{PDD} and to present the results we developed our own visualization approach, a hexa-pie-chart choropleth cartogram. This approach can present multi-variate data in a comprehensive and reasonable manner as the maps in section \ref{subsec:results_proxy_deforestation_drivers} show. However, improvements for our approach could be: modification of the pie-chart split algorithm and the representation of ratios as a piece of the hexagon. In case of the split algorithm an enhancement could be the method described by \citet{Skala1994}. A modified version of the method based on parametric separation functions could be used to generalize the pie-chart generation for the entire set of convex polygons. In case of enhancing the representation of ratios as a pieces of the hexagon we determine the piece size as a ratio of the y-range but hexagons have not an equal area under its curve at each point of y like quadrilaterals. To enhance this a scaling approach could be used, where we draw within the hexagon interior a new polygon scaled down by a ratio.

		To scrutinize how reliable our results are we performed an accuracy assessment for our \acp{PDD} mapping product derived from the \ac{GL30} and \ac{GFC} dataset. Our mapping product has an overall accuracy of approximately 75\%, while the most accurate classes in terms of producers accuracy are cropland, regrowth, grassland, and shrubland (producers accuracy ranges between 71\% and 81\%). For the remaining \ac{LC} classes forest, wetland, water, artificial, and bareland the producers accuracy ranges between 36\% and 67\%. The low accuracy within this classes can be attributed to our reclassification approach. The approach favors classes which are more frequent over the entire dataset for the reclassification. This can be explained by probability theory: if the algorithm constructs the buffer around a cluster of misclassified pixels the most dominant \ac{LC} classes are more likely to appear within the buffer. The reclassification could be tested to a nearest neighbor approach, where the reassignment is determined by the nearest neighboring cluster of a \ac{LC} class. Further, the accuracy assessment should be repeated by applying the approach described by \citet{Olofsson2014}. By applying this we would gain for each \acp{PDD} class a uncertainty assessment for the derived area estimates. This would be an improvement for the computation of derived features like the \ac{ESV} and emission estimates. However, the accuracy assessment was performed by the author, although for this kind of assessments it is critical that it is executed by independent experts.

	\section{Carbon loss}
		The second research goal is to quantify the carbon losses produced by the removal of biomass (aboveground and below-ground) and \ac{SOC} change. The total carbon loss estimates of our study should not interpreted as net carbon emissions released as \ac{GHG} to the atmosphere, but as a potential anthropogenic carbon emission source. To assess the net carbon emissions released by deforestation a more sophisticated modeling approach is required. This approach has to consider carbon pathway dynamics like carbon sequestration by new \ac{LC} types and biomass usage scenarios like wood construction, fuelwood, and timber exports.

		We quantified the total carbon losses by applying a biomass density map from \ac{GFW} and equations to convert these densities to aboveground and below-ground carbon content. At the global scale biomass removal accounts for a total carbon loss of approximately 6757 Mt C. This estimate is in line with the assessed quantities in the literature as presented in the results section \ref{sec:results_carbon_loss}. At the continental scale major carbon losses by biomass removal can be observed in Latin America. This can be attributed to the large quantities of deforested area for 2001-2010 time period. Our study demonstrates that our approach can be applied to produce reliable carbon loss estimates for the biomass removal. However, a major improvement would be the inclusion of a uncertainty assessment for the biomass density per pixel. This would enable us to provide a carbon loss range. In the future this could be included, because the next iteration of the \ac{GFW} biomass density map is scheduled with a uncertainty assessment. 

		To assess the total carbon loss by \ac{SOC} change we applied the most recent \ac{SOC} map (\ac{GSOCmap}) and \ac{SOC} change coefficients of \citet{Don2010}. At the global scale the total carbon loss accounts 583 ($\pm$ 105) Mt C or 302 ($\pm$ 76) Mt C, depending on the selected deforestation scenario. Comparable with the carbon loss by biomass removal, major losses by \ac{SOC} change can be observed in Latin America followed by Asia/Australia and Africa. We can not compare our estimates with literature values, due to the lack of studies that quantify carbon loss by \ac{SOC} change. The first \ac{SOC} change scenario (SC$_1$) assumes that all deforestation and \ac{LC} transitions occur in primary forest, while the second scenario (SC$_2$) distinguishes between primary and secondary forest transitions by applying the \ac{IFL} stratum. Due to the fact that not all tropical forest cover is primary forest, due to historic deforestation and regrowth dynamics, it can be assumed that the carbon loss estimates of the first scenario represents an overestimate. In contrast, the second scenario must be interpreted as a conservative estimate, because the \ac{IFL} stratum considers only large continues patches of intact forest landscapes as primary forest. Therefore, small patches of primary forest are not monitored for this layer. However, it is safe to assume that the total carbon loss by \ac{SOC} change ranges between both scenario estimates. In regards to the limited number of \ac{LC} transition pathways covered by the \ac{SOC} change coefficients of \citep{Don2010} we were not able to consider transitions of forest cover to artificial surfaces, which would add another quantity of carbon losses. However, our study demonstrates that carbon loss by \ac{SOC} changes adds between 3.3\%-10.2\% to the total carbon loss by tropical deforestation through \acp{PDD}. In the future the \ac{GSOCmap} will be provided to the scientific community with a per pixel uncertainty assessment. Therefore, future studies building on our method could apply this improved map to prepare \ac{SOC} loss estimates with a uncertainty range.

	\section{Ecosystem service values}
		The last research objective is to estimate the magnitudes of the ecosystem \ac{ESV} dynamics of tropical forest cover change. We relied on our \acp{PDD} mapping product to analyze the \ac{ESV} dynamics on a global and continental level by using the benefit transfer method. We derived the \ac{ESV} unit value for different biomes from three different commonly used datasets: \citet{Costanza2014} (Co), \citet{Groot2012} (Dg), and \citet{Siikamaki2015} (Wb).

		Tropical tree cover loss between 2001-2010 accounts for an \ac{ESV} gross loss of of 414.1 (Co), 402.2 (Dg), or 101.1 (Wb) billion dollar per year for the three datasets. \citet{Song2018} estimates, using Co's unit values, that tropical tree cover loss accounts for a gross loss of 550.7 billion dollar per year on a global range. The difference from our estimate can be attributed to variations in the usage of the \ac{GFC} dataset. For his study \ac{ESV} loss is computed by considering tree cover loss within the entire canopy density range. Latin America has the largest loss of \ac{ESV} by tropical deforestation followed by Asia/Australia and Africa. This can be attributed to the large deforestation in Latin America between 2001 and 2010.

		In regards of differences between the \ac{ESV} estimates of the three datasets: The unit values of \citet{Groot2012} relate to the same database as the \citet{Costanza2014} values, but only form an earlier state. Therefore, these unit values can be replaced by the \citet{Costanza2014} dataset. The Co dataset covers the most biomes and should be favored if \ac{ESV} dynamics are scrutinized for a wide variety of \ac{LC} types. The dataset of \citet{Siikamaki2015} has the smallest unit value for forest but it provides for each country a distinct estimate. The difference of the unit value for forest between Co and Wb can be attributed to the valuation of ecosystem services and the considered number of sub-services. Therefore, both estimates prepared by both datasets are not comparable.

		The \ac{ESV} net loss accounts for 47.3 (Co), 121.5 (Dg), or 43.7 (Wb) billion dollar per year on a global range. In Asia/Australia the smallest net loss is observed followed by Africa and Latin America. The difference is related to \ac{LC} transition patterns and total forest loss. In Asia/Australia most of the forest loss is driven by regrowth dynamics, while the total forest loss is smallest in Africa. To best of our knowledge no estimates for \ac{ESV} dynamics by \acp{PDD} are provided in literature for a generous comparison with our study.

		However, the net loss estimates show that the \ac{ESV} loss is by magnitude smaller if we consider the transitions of forest land to other \ac{LC} classes This reveals a fundamental problem of applying \ac{ESV} unit values to \ac{LC} change dynamics. If the entire forest cover would be replaced by cropland or artificial surfaces the \ac{ESV} dynamics would result in a net gain. Further, the coverage of different biomes by unit values is not diverse enough. In Asia/Australia and the other continents we had to use the unit value of tropical forest to compute the \ac{ESV} value of regrowth. However, regrowth frequently represents the establishment of tree crops or plantations, which does not corresponds to the \ac{ESV} of natural forests. Therefore, a greater variety of unit values is needed to cover different \ac{LC} scenarios. A future task is to consider uncertainty for the forest \ac{LC} transitions area estimates and to include these uncertainties in the \ac{ESV} estimates.