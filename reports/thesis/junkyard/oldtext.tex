
\section{Results - Forest definition}
FIGURE DESCRIPTION
	For Latin America, Asia/Australia and Africa as well the entire study extent figure \ref{fig:fordef} shows the distribution of the computed \acp{JI} for all tile pairs within the canopy density intervals. The x-axis are the different canopy density intervals where the label $JI_0$ accounts for $(0,100]$, $JI_1$ $(10,100]$, $JI_2$ $(20,100]$, and $JI_3$ $(30,100]$, respectively. The y-axis is the corresponding \ac{JI} between 0 and 1 where 0 highlights a complete disagreement and 1 a full agreement. The sample mean is labelled by a red cross and the boxes comprises the $Q_1$ (25 \%), $Q_2$ (50 \%), and $Q_3$ (75 \%) sample interval, respectively.

OLD INTRO
	Our goal is to determine at which canopy cover density the agreement between \ac{GL30} and \ac{GFC} tree cover is greatest to receive the subsequent \ac{PDD} for stable \ac{LC} transitions introduced by anthropogenic causes. This process should ensure that we keep the largest number of tree cover loss samples from the \ac{GFC} dataset while harmonizing the tree cover definition between both layers. We applied the \ac{JI} to determine the similarity between each tile pair from our \ac{AISM}. The \ac{JI} computation is grouped by the continental regions Latin America (82 tiles), Asia/Australia (86 tiles), and Africa (101 tiles). We determined the similarity for the following canopy density intervals: $(0, 100]$, $(10, 100]$, $(20, 100]$, and $(30, 100]$. Later we excluded all tiles with a initial \ac{JI} (canopy density interval $(0,100]$) from our analysis because these tile pair does not contain any tree cover. We excluded 6, 9, and 15 tiles for Latin America, Asia/Australia and Africa, respectively. To determine the canopy density interval where the agreement is at maximum we applied the non-parametric tests Wilcoxon signed-rank test and Wilcoxon rank-sum test. Both of the tests are performed as a one- and two-sided to deduce, if there is a difference in agreement (equality) and which direction (less or greater) has this difference. To address the higher probability of family-wise error rate in multiple comparisons we applied a Holm correction for Wilcoxon signed-rank tests and a Benjamini and Hochberg correction for Wilcoxon rank-sum test. We applied continental and global testing to deduce regional differences and to determine the optimum for our subsequent \ac{PDD} predictions. Further, we compared the tree cover agreement of the three continental regions. Our initial hypothesis was that the tree cover agreement is at its maximum within the canopy density interval of $(30,100]$ for the entire study extent. We assumed that for Latin America and Asia/Australia the best results could be achieved with the same canopy density threshold. For Africa we assumed that the highest agreement could be achieved within the interval of $(10,100]$ because this region comprises a higher frequency of sparse woodland cover. The following paragraphs present our results for the three regions in the following order: Latin America, Asia/Australia and Africa. The last paragraph discusses the results for the entire study extent and determines which canopy density we used for the following \ac{PDD} prediction.

\section{Results - Tree cover and deforestation patterns}
OLD INTRO
	This section is intended to present a comprehensive insight in the tropical tree cover distribution within our study extent at 2000 over the tree continental regions. Further, we highlight at which sites the tree cover loss peaked between 2001 and 2010. The tree cover maps are derived from the \ac{GFC} tree cover 2000 layer while we selected only pixels within our forest definition which refers to the canopy density interval $(10,100]$. For an appropriate visualization of multivariate spatial data on a large extent we selected our hexagonal binning approach. For the tree cover maps we computed the total area covered by trees within a polygon and divide by the total area of the hexagon to determine the scaling. Additionally we aggregate the canopy density within a hexagon by applying the arithmetic mean. To arrange the tree cover loss maps we used our \ac{PDD} products. We computed the loss area within each hexagon for the following \ac{PDD} classes: cultivated land (10), regrowth (25), grassland (30), shrubland (40), artificial surfaces (80), and bareland (90). To determine the polygon scaling we divided the per hexagon loss area by the highest observed loss within a continental region. Forest cover, losses, and hexagon areas are computed by applying the Haversine equation. A hexagon in unscaled shape covers an area of 0.5 decimal degrees. The maps in this section should be interpreted as precursor to our \ac{PDD} predictions to detect regional and continental patterns of deforestation and as an example how large multivariate spatial data can be visualized and evaluated by a more advanced aggregation approach.

\section{Results - }
OLD INTRO
	Our goal is to estimate the distribution of \acp{PDD} over the tropical zone, the continental range, and at a country scale for the time frame of 2001 till 2010. We achieved this estimate by superimposing the annual tree cover losses and aggregated gains of the \ac{GFC} datasets and the \ac{GL30} land cover map from 2000. We carefully selected our global definition of tree cover in the canopy density interval $(10,100]$ by applying the Jaccard Index and statistical testing detailed in section \ref{subsec:results_forest_definition}. By using this canopy density interval we filtered the \ac{GFC} annual losses and we considered tree cover gains only within previously lost tree cover. After superimposing we applied a reclassification of tree cover losses still classified as forest by the \ac{GL30} layer. We aggregated this structures by clustering and applied a square sized buffer of 500 meter side length. Next, we reclassified the structure by determining the highest frequent \ac{LC} class within the buffer. For further details on the \ac{PDD} prediction refer to section \ref{subsubsec:methods_proximate_deforestation_driver}. The \ac{PDD} distribution choropleth-maps for Latin America, Asia/Australia, and Africa we derived by applying our hexagonal-binning approach in combination with a hexagon-pie-chart. A hexagon in unscaled shape covers an area of 0.5 decimal degrees. Section \ref{subsec:methods_binning} describes detailed how we derived these cartograms.

\section{Results - Ecosystem service values}
OLD INTRO
	Our goal is to estimate the monetary loss of \ac{ESV} by the aggregated \ac{PDD} at a global and continental scale between 2001 and 2010. Whereas a study targeting regional scale aggregation could easily performed with our data. Additionally, we approximated the monetary gain if the former tree cover is converted to a certain type of \ac{LC}. We refer to this gain as \ac{ESV} gain. Further, we compute the balance between \ac{ESV} loss and gain to estimate the net change of \ac{ESV} in the tropical zone. We applied three datasets which estimate the monetary value of ecosystems on a global scale namely the following data: \citet{Groot2012}, \citet{Costanza2014}, and \citet{Siikamaki2015}. We refer to this datasets as Co, Dg, and Wb. To compute the monetary loss of \ac{ESV} we selected the following \ac{PDD} classes as anthropogenic deforestation: cultivated land (10), regrowth (25), grassland (30), shrubland (40), artificial surfaces (80), and bareland (90). We excluded pixel classified as forest (20) by our \ac{PDD} prediction because within this class we are uncertain if a deforestation event occurred. Further, we excluded transitions to wetland or water bodies because we assume this \ac{LC} changes are largely driven by natural causes. The cumulative \ac{ESV} gain depends on the particular \ac{ESV} dataset because the datasets don't define a monetary value for each \ac{PDD} class. For detailed informations on the methodology refer to section \ref{subsec:methods_esv}. The \ac{ESV} net change or balance is the difference of cumulative \ac{ESV} loss and gain. The monetary unit for each value is the Geary–Khamis dollar at 2007 per year also known as international dollar (2007 Int.\$ y$^{-1}$). The table \ref{tab:esv_results} shows the gross \ac{ESV} loss and gain and the net balance for continental and global scale. We will discuss the \ac{ESV} losses exclusively for the Co and Wb datasets because the difference between Co and Dg is just 118 international dollar per year. Further, we discuss the \ac{ESV} gain for all three datasets because they provide a range of different estimates per biome.

NEW INTRO
	We scrutinized \ac{ESV} dynamics of tropical forest cover change by applying benefit transfer for the period 2001-2010. In this section we will present the results of our study. First we will discuss the \ac{ESV} dynamics on global scale followed by the discussion of the continental regions Latin America, Asia/Australia, and Africa.